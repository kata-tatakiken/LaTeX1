\documentclass[paper=a4paper,fontsize=10pt]{jlreq}
\usepackage{luatexja-fontspec}
\setmainfont{Harano Aji Mincho}
\setsansfont{Harano Aji Gothic}
\setmainjfont{Harano Aji Mincho}
\setsansjfont{Harano Aji Gothic}

\usepackage{amsmath,amssymb}
\usepackage{unicode-math}
\setmathfont{LatinModernMath-Regular}

\begin{document}

\title{タイトル}
\author{私 太郎}
\date{\today}
\maketitle

\tableofcontents

\part{これが部}
\section{これが節}
\subsection{これが小節}
\subsubsection{これが小小節}
\paragraph{これは段落}
\subparagraph{これこそが小段落}
こんな感じで文章構造を組むことができます。ちなみに、フォントを途中で変えることもできます。\gtfamily ここはゴシック体、\rmfamily This is serif, \sffamily This is sans-serif, \ttfamily This is monospaced.\rmfamily \mcfamily 事前に読み込んでいないフォントも読み込めます。\fontspec{BIZ UDP Gothic}\jfontspec{BIZ UDP Gothic}ここは、事前に読み込んでいない、BIZ UDPゴシックです。\rmfamily\mcfamily\bfseries{こんな感じで太字になります。}\mdseries 斜体とかは、めんどくさいのと僕がちゃんと理解してないので説明しません。あんまり斜体ってつかわないよね。\\
こんな感じで強制改行も行えます。

\section{特殊文字とか}
\mathrm{\LaTeX} とかの特殊な書き方はこんなふうに表示できます。\\
音楽記号\sharp \ \flat とかもあります。

\section{数式も埋め込めます}
$y = x$みたいな感じで、行中に埋め込めますし、以下のように書くこともできます。\\

\begin{equation}
  n\longrightarrow 0 のときa_n\longrightarrow \infty
\end{equation}

複数行の数式も書けます。$=$の位置を揃えることもできます。\\

\begin{align}
  \int_{1}^{2}\left(x^2 + 3x\right)dx + \int_{1}^{2}\left(x^2 - 3x\right)dx &= \int_{1}^{2}\left\{\left(x^2 + 3x\right) + \left(x^2 - 3x\right)\right\}dx \\
  &= \int_{1}^{2}2x^2dx \\
  &= 2\left[\frac{x^3}{3}\right]^2_1 = \frac{2\left(2^3 - 1^3\right)}{3} = \frac{14}{3}
\end{align}

\end{document}
